%!TEX program = xelatex
\documentclass{article}
\usepackage[a5paper,hmargin=17mm,tmargin=15mm,bmargin=25mm]{geometry}

\usepackage{ifxetex}
\ifxetex
 \usepackage{fontspec}
 \setmainfont[Scale=1.1]{Arno Pro}
 \setmonofont[Scale=.92]{Consolas}
 \usepackage{unicode-math}              %% пакет для загрузки шрифтов математического режима 
 \setmathfont{[latinmodern-math.otf]}
 \setmathfont[range=\mathit/{latin,Latin}]{Arno Pro Italic}
 \setmathfont[range=up]{Arno Pro}
 \setmathfont[range=\mathup/{latin,Latin}]{Arno Pro}
\else
 \usepackage[utf8]{inputenc}
\fi
\usepackage[russian]{babel}
\usepackage{enumitem}


\begin{document}
\section*{{\normalsize Лабораторная работа 3}\\Циклы}

Цель этой лабораторной работы~--- изучить понятие цикла и~научиться записывать различные циклы в языке Си. 

\bigskip\sloppy
\noindent\centerline{\textbf{ОБЩИЕ ЗАДАНИЯ}}
\begin{enumerate}
\item 
Напишите программу, которая вводит с клавиатуры 10 целых чисел и выводит их среднее арифметическое.
\item
Напишите программу, которая вводит с клавиатуры число $n$ от 1 до 20 включительно и печатает пирамиду из решеток и пробелов высоты $n$, например для $n=3$ программа должна напечатать три строки:\\
\verb!  #!\\
\verb! ###!\\
\verb!#####!
\item
Напишите программу, которая вводит с клавиатуры целое число типа \texttt{long long} (в \texttt{scanf} указывать код формата \texttt{\%lld}) и выводит его цифры в обратном порядке.
\item
Для быстрой офлайновой (без связи с банком) проверки правильности написания номера кредитной карты используется алгоритм Х.\,П.~Луна: умножим, двигаясь справа налево, каждую вторую цифру номера на 2. Сложим все цифры полученных чисел (внимание, не сами числа!). Теперь прибавим к ним сумму остальных цифр. Если последняя цифра полученной общей суммы не ноль, номер неправильный. Пример: номер \underline{4}3\underline{7}2\,\underline{2}8\underline{2}2\,\underline{4}4\underline{3}1\,\underline{0}0\underline{0}5 верный: удвоения подчеркнутых цифр равны 8, 14, 4, 4, 8, 6, 0, 0, их цифры в сумме дают $8 + (1\!+\!4) + 4 + 4+ 8+ 6 + 0+ 0 = 35$, сумма неподчеркнутых цифр номера равна $3+2+8+2+4+1+0+5=25$, а $35+25=6\mbox{\textbf{\textit{0}}}$.

Напишите программу, которая вводит номер карты \underline{как целое число} (тип long long, форматный спецификатор \texttt{\%lld}, \textbf{НЕ} как строку) и печатает \texttt{VALID} или \texttt{INVALID}, если номер, соответственно, верный или неверный по алгоритму Луна. Проверки будут производиться на числах, содержащих от 13 до 16 цифр. 

\item
К волшебному пределу $e = \lim\limits_{n\to\infty}(1+\frac{1}{n})^n$ (<<два и семь
и дважды год рождения Тостого>>) можно приблизиться по-другому: ряд 
$$ 1 + x + \frac{x^2}{2!} + \frac{x^3}{3!} + \ldots +  \frac{x^n}{n!}+\ldots \eqno{(*)}$$
сходится к $e^x$ при любых $x$.

Напишите программу, которая вводит с клавиатуры вещественное $x$ (используйте тип \texttt{double}), double), вычисляет для данного $x$ частичную сумму ряда $(*)$, в которой участвуют только те слагаемые ряда, которые не меньше 0.0001 (слагаемые монотонно убывают), и выводит модуль разности между вычисленной суммой и значением библиотечной функции \texttt{exp(x)} из заголовочного файла \texttt{cmath}/\texttt{math.h}  (вывести не менее 6 знаков после десятичной точки).

\end{enumerate}



\begin{center}
\bf\Large  Домашнее задание 3
\end{center}

\noindent 
\textbf{Всего в этом домашнем задании 5 пунктов. Из каждого пункта надо сделать по одной задаче, в зависимости от вашего номера в списке группы. Считать по кругу: если в пункте 6 задач, то 7-й номер делает первую задачу, 8-й номер вторую и т.д.}

\medskip
\noindent 

\subsection*{Задачи}
\begin{enumerate}[label={}, leftmargin=0pt, itemindent=0pt]
\item
\begin{enumerate}[label=\arabic{enumi}.\arabic*.]
\item
Напишите программу, которая вводит целое число $n$, $1\leqslant n \leqslant 20$, и выводит текстовое поле размера $n\times n$ из точек и решеток в шахматном порядке, причем в левом верхнем углу находится решетка.
Например, для $n=5$ вывести\\
\verb!#.#.#!\\
\verb!.#.#.!\\
\verb!#.#.#!\\
\verb!.#.#.!\\
\verb!#.#.#!
\item 
Напишите программу, которая вводит целое число $n$, $1\leqslant n \leqslant 20$, и выводит текстовое поле размера $n\times n$ из точек и решеток в шахматном порядке, причем в левом верхнем углу находится точка.
Например, для $n=5$ вывести\\
\verb!.#.#.!\\
\verb!#.#.#!\\
\verb!.#.#.!\\
\verb!#.#.#!\\
\verb!.#.#.!
\item 
Напишите программу, которая вводит целое число $n$, $2\leqslant n \leqslant 20$, и выводит рамку размера $n\times n$ из решеток. Например, для $n=4$ вывести\\
\verb!#### !\\
\verb!#  # !\\
\verb!#  # !\\
\verb!#### !
\item 
Напишите программу, которая вводит целое число $n$, $2\leqslant n \leqslant 20$, и выводит контур треугольника высоты $n$ с основанием ширины $2n$. Основание изображается подчеркиванием, стороны~--- прямыми и обратными косыми чертами. Например, для $n=5$ вывести
\\
\verb!    /\!\\
\verb!   /  \!\\
\verb!  /    \!\\
\verb! /      \!\\
\verb!==========!
\end{enumerate}
\hrulefill

\item
\begin{enumerate}[label=\arabic{enumi}.\arabic*.]
\item
Напишите программу, которая вводит 10 целых чисел и печатает через пробел количество положительных, отрицательных и нулей среди них. Например для ввода \texttt{ 1 -2 5 3 0 -3 -3 7 8 -1} вывести \texttt{5 4 1}.
\item
Напишите программу, которая вводит 10 неотрицательных чисел и печатает максимум среди четных из них, либо -1, если четных чисел нет. Например для ввода \texttt{ 1 2 5 3 0 3 0 7 8 9} вывести \texttt{8}.
\item
Напишите программу, которая вводит 10 неотрицательных чисел и печатает максимум среди нечетных из них, либо -1, если нечетных чисел нет. Например для ввода \texttt{ 1 2 5 3 0 3 0 8 4 7} вывести \texttt{7}.
\item
Напишите программу, которая вводит 10 неотрицательных чисел и печатает число таких, которые не меньше своих соседей. Например для ввода \texttt{ 1 2 5 3 0 4 0 7 8 9} вывести \texttt{3} (числа 5, 4, 9).
\item
Напишите программу, которая вводит 10 целых чисел и печатает число с максимальным модулем. Например для ввода \texttt{ 1 -2 5 3 0 -3 -3 7 -8 -1} вывести \texttt{-8}.
\end{enumerate}

\hrulefill
\item
\begin{enumerate}[label=\arabic{enumi}.\arabic*.]
\item
Натуральное число из $n$ цифр называется числом Армстронга, если сумма его цифр, возведенных в $n$-ю степень, равна самому числу (например $153=1^3+5^3+3^3$). Напечатать в возрастающем порядке все числа Армстронга, состоящие из четырех цифр.
\item
Числа Фибоначчи определяются формулами:
$F_0=0$; $F_1=1$; $F_n=F_{n-1}+F_{n-2}$, $n=2,3,\dots$. 
Ввести с клавиатуры целое число $M$ и вывести через пробел первое число Фибоначчи, большее $M$, и его номер $k$. Например, для $M=5$ следует вывести \texttt{8 6}.
\item
Даны натуральные числа $m$ и $n$. Найти наименьшее общее кратное этих чисел. Например, для входных данных \texttt{6 20} вывести \texttt{60}.
\item
Ввести с клавиатуры целое положительное число $k$ и напечатать $k$-е простое число. Например, для $k=5$ вывести \texttt{11}.
\item
Дано целое число $n\geqslant 0$. Вывести сумму его цифр.
\item
Найти наибольшую и наименьшую цифры в записи данного натурального числа $n$.
\end{enumerate}
\hrulefill


\item

\begin{enumerate}[label=\arabic{enumi}.\arabic*.]
\item
Напишите программу, которая вводит с клавиатуры четыре числа $d_1$, $m_1$, $d_2$, $m_2$~--- число и месяц для двух дат одного и того же невисокосного года, причем вторая дата не раньше первой. Программа должна выводить одно число~--- на сколько дней вторая дата позже первой. Например, для входных данных \texttt{1 3 8 3} вывести \texttt{7}, для входных данных \texttt{1 1 31 12} вывести \texttt{364}.
\item
Напишите программу, которая вводит с клавиатуры три целых числа $d$, $m$, $k$~--- число и месяц, а также количество дней. Программа должна выводить два числа~--- число и месяц, наступающий через $k$ дней после даты $d/m$. Гарантируется, что вторая дата в том же году, что и первая. Например, для входных данных \texttt{1 9 45} вывести \texttt{16 10}, так как через 45 дней после 1~сентября наступает 16~октября.
\item
Напишите программу, которая вводит с клавиатуры три целых числа $d$, $m$, $k$~--- число и месяц, а также количество дней. Программа должна выводить два числа~--- число и месяц бывший $k$ дней назад относительно даты $d/m$. Гарантируется, что вторая дата в том же году, что и первая. Например, для входных данных \texttt{1 9 92} вывести \texttt{31 5}, так как за 92~дня до 1~сентября было 31~мая.
\item
Напишите программу, которая вводит через пробел целые числа $K$, $s$, и вещественное число $p$. $K$ является первоначальным размером кредита, выданного под $p$~\% годовых. Кредит погашается равными платежами по $s$ рублей каждый месяц. Из этой суммы сначала уплачивается проценты за пользование кредитом в размере $\frac{p}{12}$ процентов от текущей суммы долга, а затем оставшаяся часть суммы  направляется на уменьшение суммы долга. Программа должна выводить по одному в строке оставшуюся сумму долга на конец каждого месяца до полного погашения кредита. Последним числом в выводе должен быть ноль. Например, при сумме долга $K=100000$, ставке $p=12$\,\%, и сумме ежемесячного платежа $10000$, должны быть распечатаны числа\\
\texttt{
91000.00\\
81910.00\\
72729.10\\
63456.39\\
54090.95\\
44631.86\\
35078.18\\
25428.96\\
15683.25\\
5840.09\\
0.00}
\item
Напишите программу, которая вводит с клавиатуры номер $n$ дня в году (от 1 до 365) и выводит день и месяц в виде двух чисел (например, для $n=33$ вывести \texttt{2 2}, так как 33-й день года~--- 2 февраля).

\end{enumerate}
\hrulefill

\item
\begin{enumerate}[label=\arabic{enumi}.\arabic*.]
\item 
Выведите по одному в строке все трехзначные простые числа.
\item 
Введите с клавиатуры число $N>5$. Натуральные числа $a$, $b$, $c$ называются числами Пифагора, или пифагоровой тройкой, если выполняется условие $a^2+b^2=c^2$. Напечатать все пифагоровы тройки, в которых все числа меньше $N$. Каждую в отдельной строке, числа в каждой строке в возрастающем порядке через пробел. Например, для $N=6$ напечатать \texttt{3 4 5}.
\item 
Даны через пробел натуральные числа $n$, $m$. Напечатать по одному в строке в возрастающем порядке все натуральные числа, меньшие $n$, квадрат суммы цифр которых равен $m$.
Если таких нет, напечатать \texttt{-1}. Например для входных данных \texttt{100 4} напечатать числа \texttt{2}, \texttt{11}, \texttt{20}, каждое в отдельной строке.
\item
Последовательность Хэмминга образуют натуральные числа, не имеющие других простых делителей, кроме $2$, $3$ и $5$. Введите с клавиатуры число $N$ и выведите сумму первых $N$ элементов последовательности Хэмминга. Например, для $N=8$ вывести \texttt{38} (так как $1+2+3+4+5+6+8+9=38$).
\end{enumerate}

\end{enumerate}

\end{document}
